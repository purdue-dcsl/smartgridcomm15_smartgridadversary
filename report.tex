\documentclass{article}

\usepackage{float}
\usepackage{multicol}
\usepackage{graphicx}
\usepackage{url}
\usepackage{hyperref}
\usepackage{epstopdf}

\begin{document}

\title{Thoughts about Stuff}

\author{Nathan Burow, Saurabh Bagchi}

\maketitle
\begin{multicols}{2}

Abstract goes here, especially once I get this in a better / more rigorous format.  Bare bones for now.

\section{Introduction}

It is trite but true that the power grid underlies most of modern civilization.  Traditionally the power grid has had a 
few fixed producers that sell power to utilities, who in turn distribute it to their customers.  Traditionally these 
customers have not coordinated with eachother, but simply bought whatever power they need at the prevailing price.  
Similarily, utilities have simply ensured that they have enough power to meet their customer's demands.  

The future will be very different.  More producers will enter the grid as renewables like wind and solar become more
prevalent.  Increased connectivity through the Smart Grid will allow consumers to be price sensitive, and put off 
demand during peak usage (and thus peak price) hours where possible.  This will smooth the demand curve that utilities face,
and make the grid more predictable.

The benefits of the Smart Grid rely on making the power-grid a cyber-physical system (CPS) by overlaying a cyber 
communication and control network.  It is tautological that any communications network can and will be attacked.  
We propose a strategic adversary with strong abilities, and model the adversary's interaction with the grid as a game.
We present optimal strategies for the attacker and defender under different assumptions and show .... (not sure yet).

\section{Model}

\subsection{Producer}

We focus on renewable energy sources, which provide a fluctuating amount of power.  We model them as exponential random 
number generators [justification?, other possiblities: uniform, power-law].

\subsection{Consumer}

Following [citation here] we have three classes of consumer: buckets, batteries, and bakeries.  

\subsection{Utility}

The utility's goal is to dispatch all power generated in a given time period.  Excess power is wasted, and shortfalls are 
made up from expensive standby generators.  

\subsection{Topology}

Power-law topology for consumers / producers.  We may want to dodge this.

\section{Game}

Our game is played between a strategic adversary that seeks to maximally disrupt the grid, within the constraints of his
resources and the utility company whose objective was given in [utility subsection].  

\subsection{Adversary}

We consider three different capability sets for the adversary:

\begin{itemize}
\item The attacker can jam communication between the utility and producers / consumers.  We consider scenarios where the
attacker can jam 10\%, 30\%, 60\%, and 90\% of communication links with producers / consumers.
\item Attacker can modify the communicated demand from consumers / supply from producers.  Here the attacker is constrained
to a small change in order to remain undetected. This is a false data injection attack.
\item  Attacker can change the consumer's time constraints (effectively change their class).
\end{itemize}

The attackers optimization function is the inverse of the utilities: trying to maximize instead of minimize.

\subsection{Utility}

Utility defaults to base state for each consumer if communication fails.  Conceptually, this is their demand at the same 
point the prior day.

Utility seeks to minimize power wastage / excess demand.

Utility can schedule the known demand of consumers in order to minimize this wastage.  What scheduling strategies
should the utility use, is this something we want to get into?

What is the Utilities defensive strategy: detect? If detect, can it do better than fall back to default?

\section{Experimental Results}
\section{Related Work}
\section{Conclusions}
\end{multicols}
\end{document}
