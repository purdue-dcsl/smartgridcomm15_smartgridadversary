\documentclass{article}

\usepackage{float}
\usepackage{multicol}
\usepackage{graphicx}
\usepackage{url}
\usepackage{hyperref}
\usepackage{epstopdf}

\begin{document}

\title{Thoughts about Stuff}

\author{Nathan Burow, Saurabh Bagchi}

\maketitle
\begin{multicols}{2}

Abstract goes here, especially once I get this in a better / more rigorous format.  Bare bones for now.

\section{Introduction}

It is trite but true that the power grid underlies most of modern civilization.  Traditionally the power grid has had a 
few fixed producers that sell power to utilities, who in turn distribute it to their customers.  Traditionally these 
customers have not coordinated with eachother, but simply bought whatever power they need at the prevailing price.  
Similarily, utilities have simply ensured that they have enough power to meet their customer's demands.  

The future will be very different.  More producers will enter the grid as renewables like wind and solar become more
prevalent.  Increased connectivity through the Smart Grid will allow consumers to be price sensitive, and put off 
demand during peak usage (and thus peak price) hours where possible.  This will smooth the demand curve that utilities face,
and make the grid more predictable.

The benefits of the Smart Grid rely on making the power-grid a cyber-physical system (CPS) by overlaying a cyber 
communication and control network.  It is tautological that any communications network can and will be attacked.  
We propose a strategic adversary with strong abilities, and model the adversary's interaction with the grid as a game.
We present optimal strategies for the attacker and defender under different assumptions and show .... (not sure yet).

\section{Design}
\section{Implementation}
\section{Experimental Results}
\section{Related Work}
\section{Conclusions}
\end{multicols}
\end{document}
